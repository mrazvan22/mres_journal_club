\documentclass[11pt,a4paper,oneside]{report}

\usepackage{amsmath,amssymb,calc,ifthen}

\usepackage{float}

\usepackage[table,usenames,dvipsnames]{xcolor} % for coloured cells in tables

\usepackage{tikz}


\usepackage{hyperref}
\hypersetup{
    colorlinks,
    citecolor=black,
    filecolor=black,
    linkcolor=black,
    urlcolor=black
}

% Nice package for plotting graphs
% See excellent guide:
% http://www.tug.org/TUGboat/tb31-1/tb97wright-pgfplots.pdf
\usetikzlibrary{plotmarks}
\usepackage{amsmath,graphicx}
\usepackage{epstopdf}
\usepackage{caption}
\usepackage{subcaption}

% highlight - useful for TODOs and similar
\usepackage{color}
\newcommand{\hilight}[1]{\colorbox{yellow}{#1}}

\newcommand\ci{\perp\!\!\!\perp} % perpendicular sign
\newcommand*\rfrac[2]{{}^{#1}\!/_{#2}} % diagonal fraction

\usepackage{listings}



% margin size
\usepackage[margin=2.5cm]{geometry}

\tikzstyle{state}=[circle,thick,draw=black, align=center, minimum size=2.1cm,
inner sep=0]
\tikzstyle{vertex}=[circle,thick,draw=black]
\tikzstyle{terminal}=[rectangle,thick,draw=black]
\tikzstyle{edge} = [draw,thick]
\tikzstyle{lo} = [edge,dotted]
\tikzstyle{hi} = [edge]
\tikzstyle{trans} = [edge,->]


\definecolor{mygreen}{rgb}{0,0.6,0}
\definecolor{mygray}{rgb}{0.5,0.5,0.5}
\definecolor{mymauve}{rgb}{0.58,0,0.82}

\lstset{ %
  backgroundcolor=\color{white},   % choose the background color; you must add 
%\usepackage{color} or \usepackage{xcolor}
  basicstyle=\footnotesize,        % the size of the fonts that are used for the 
%code
  breakatwhitespace=false,         % sets if automatic breaks should only happen 
%at whitespace
  breaklines=true,                 % sets automatic line breaking
  captionpos=b,                    % sets the caption-position to bottom
  commentstyle=\color{mygreen},    % comment style
  deletekeywords={...},            % if you want to delete keywords from the 
%given language
  escapeinside={\%*}{*)},          % if you want to add LaTeX within your code
  extendedchars=true,              % lets you use non-ASCII characters; for 
%8-bits encodings only, does not work with UTF-8
  frame=single,                    % adds a frame around the code
  keepspaces=true,                 % keeps spaces in text, useful for keeping 
%indentation of code (possibly needs columns=flexible)
  keywordstyle=\color{blue},       % keyword style
  language=Octave,                 % the language of the code
  morekeywords={*,...},            % if you want to add more keywords to the set
  numbers=left,                    % where to put the line-numbers; possible 
%values are (none, left, right)
  numbersep=5pt,                   % how far the line-numbers are from the code
  numberstyle=\tiny\color{mygray}, % the style that is used for the line-numbers
  rulecolor=\color{black},         % if not set, the frame-color may be changed 
%on line-breaks within not-black text (e.g. comments (green here))
  showspaces=false,                % show spaces everywhere adding particular 
%underscores; it overrides 'showstringspaces'
  showstringspaces=false,          % underline spaces within strings only
  showtabs=false,                  % show tabs within strings adding particular 
%underscores
  stepnumber=2,                    % the step between two line-numbers. If it's 
%1, each line will be numbered
  stringstyle=\color{mymauve},     % string literal style
  tabsize=2,                       % sets default tabsize to 2 spaces
  title=\lstname                   % show the filename of files included with 
%\lstinputlisting; also try caption instead of title
}

%\title{EPSRC Centre for Doctoral Training in Medical Imaging\\\hfill\\MRes Project Plan}
% \title{Differential Diagnosis of Alzheimer subtypes through disease progression modelling}
% \author{
% Razvan Valentin Marinescu\\
% Student Number: 14060166\\
% \texttt{razvan.marinescu.14@ucl.ac.uk}
% }

\begin{document}
\belowdisplayskip=12pt plus 3pt minus 9pt
\belowdisplayshortskip=7pt plus 3pt minus 4pt


\begin{titlepage}
\begin{center}

% Upper part of the page. The '~' is needed because \\
% only works if a paragraph has started.
\includegraphics[width=0.2\textwidth]{ucl-logo2}~\\[1cm]


\textsc{\Large CDT Journal Club - First Written Review}\\[0.5cm]

\newcommand{\HRule}{\rule{\linewidth}{0.5mm}}

% Title
\HRule \\[0.4cm]
{ \Large Automatic classification of MR scans in Alzheimer's disease \\[0.4cm] }

{ \small Stefan Klöppel , Cynthia M. Stonnington , Carlton Chu , Bogdan Draganski , Rachael I. Scahill , Jonathan D. Rohrer , Nick C. Fox , Clifford R. Jack , John Ashburner , Richard S. J. Frackowiak}

\HRule \\[1.5cm]

% Author and supervisor
\begin{minipage}{0.4\textwidth}
\begin{flushleft} \large
\emph{Review Author:}\\
R\u{a}zvan Valentin \textsc{Marinescu}\\
\href{razvan.marinescu.14@ucl.ac.uk}{razvan.marinescu.14@ucl.ac.uk}\\
\end{flushleft}
\end{minipage}
\begin{minipage}{0.4\textwidth}
\begin{flushright} \large
\emph{Paper chosen by:} \\
Prof. Sebastien \textsc{Ourselin}\\
\end{flushright}
\end{minipage}

\vfill

EPSRC Centre for Doctoral Training in Medical Imaging\\ University College London

\vfill

% Bottom of the page
{\large \today}

\end{center}
\end{titlepage}
% \maketitle{}


\section*{Paper background}

Dementia and associated diseases such as Alzheimer's disease (AD) affect 36 million people globally \cite{world2012dementia} and account for 486,000 deaths \cite{lozano2013global}. Definitive diagnosis of AD can only be made with histopathological confirmation of amyloid plaques and neurofibrillary tangles, normally at autopsy. Early accurate detection of AD is important because the treatment is most effective if undertaken as early as possible. Medical imaging such as MRI has been used as a tool for AD diagnosis by measuring the atrophy rates of cortical volumes \cite{fox2004imaging,barnes2004differentiating} but these methods have not yet been introduced in clinical practice.

\section*{Introduction}



\section*{Method}

\section*{Key results}

\section*{Contributions}

\section*{Limitations}

\section*{Impact}


\nocite{*} % Show all Bib-entries
\bibliographystyle{unsrt}
\bibliography{citations}


\end{document}
