\documentclass[11pt,a4paper,oneside]{report}

\usepackage{amsmath,amssymb,calc,ifthen}

\usepackage[table,usenames,dvipsnames]{xcolor} % for coloured cells in tables

\usepackage{hyperref}


\usepackage{amsmath,graphicx}
\usepackage{epstopdf}
\usepackage{caption}
\usepackage{subcaption}

% highlight - useful for TODOs and similar
\usepackage{color}


\usepackage{listings}

\hypersetup{
  colorlinks,
  citecolor=Blue,
  linkcolor=Red,
  urlcolor=Blue}
  

% margin size
\usepackage[margin=2.5cm]{geometry}



\definecolor{mygreen}{rgb}{0,0.6,0}
\definecolor{mygray}{rgb}{0.5,0.5,0.5}
\definecolor{mymauve}{rgb}{0.58,0,0.82}


\begin{document}
\belowdisplayskip=12pt plus 3pt minus 9pt
\belowdisplayshortskip=7pt plus 3pt minus 4pt


\begin{titlepage}
\begin{center}

% Upper part of the page. The '~' is needed because \\
% only works if a paragraph has started.
\includegraphics[width=0.2\textwidth]{ucl-logo2}~\\[1cm]


\textsc{\Large CDT Journal Club - Second Written Review}\\[0.5cm]

\newcommand{\HRule}{\rule{\linewidth}{0.5mm}}

% Title
\HRule \\[0.4cm]
{ \Large [18F]T807, a novel tau positron emission tomography imaging agent for Alzheimer's disease \\[0.4cm] }

{ \small Chun-Fang Xia, Janna Arteaga, Gang Chen, Umesh Gangadharmath, Luis F. Gomez, Dhanalakshmi Kasi, Chung Lam, Qianwa Liang, Changhui Liu, Vani P. Mocharla, Fanrong Mu, Anjana Sinha, Helen Su, A. Katrin Szardenings, Joseph C. Walsh, Eric Wang, Chul Yu, Wei Zhang, Tieming Zhao, Hartmuth C. Kolb }

\HRule \\[1.5cm]

% Author and supervisor
\begin{minipage}{0.4\textwidth}
\begin{flushleft} \large
\emph{Review Author:}\\
R\u{a}zvan Valentin \textsc{Marinescu}\\
\href{razvan.marinescu.14@ucl.ac.uk}{razvan.marinescu.14@ucl.ac.uk}\\
\end{flushleft}
\end{minipage}
\begin{minipage}{0.4\textwidth}
\begin{flushright} \large
\emph{Paper lead by:} \\
Prof. Erik \textsc{Arstad}\\
\end{flushright}
\end{minipage}

\vfill

EPSRC Centre for Doctoral Training in Medical Imaging\\ University College London

\vfill

% Bottom of the page
{\large \today}

\end{center}
\end{titlepage}
% \maketitle{}




\section*{Aims and motivations}

Tau-related neurological disorders such as Alzheimer's disease (AD), Frontotemporal lobe degeneration (FTLD), Down's syndrome and Progressive supranuclear palsy (PSP) affect more than 33.9 million people worldwide. \cite{barnes2011projected} Although many amyloid-beta (A$\beta$) tracers have been developed such as Pittsburgh compound B (PiB) \cite{klunk2004imaging}, only a few tracers have been reported previously for the Tau protein, such as 18F-FDDNP\cite{barrio1999pet}, 11C-lansoprazole\cite{rojo2010selective}, 18F-THK523\cite{fodero201118f}. These had several limitations: 18F-FDDNP showed limited binding to both NFTs and A$\beta$ plaques in vitro \cite{thompson2009interaction}, 11C-lansoprazole was initially unable to cross the brain-blood barrier \cite{shao2012evaluation} while 3H-THK523 was shown to bind to neurofibrillary tangles (NFTs) as well as to A$\beta$ plaques in human AD brains \cite{zeng20128invitro}. The aim of the paper was to design and develop a novel positron emission tomography (PET) imaging agent targeting paired-helical filament (PHF) tau in human AD brains that would overcome the limitations of previous compounds.

\section*{Methods}

The authors designed a novel class of 5H-pyrido-indoles and examined their binding affinity towards PHF tau and selectivity over A$\beta$. A total of 45 AD and non-AD human brains were used and immunohistochemically (IHC) stained for determining PHF-tau and A$\beta$ load. After a qualitative screening of several hundred compunds, T807 emerged as the lead candidate. Aster several preparation steps, T807 was labelled with 18F radioisotope by heating the mixture of the two compounds to 130 $^{\circ}$C so that the -NO$_2$ group would be substituted by 18F fluoride. In order to determine the lipophilicity of the new tracer, the logP was determined by analysing it in a gamma counter. 

The 18F-T807 tracer was further analysed with regards to how well it binded to PHF-tau and A$\beta$, brain uptake and biostability. Selectivity for PHF tau over A$\beta$ was measured using autoradiography on the 26 AD and non-AD human brains. The dissociation constant Kd was determined by Scatchard plot analysis. Brain uptake, biodistribution, excretion and biostability of 18F-T807 were analysed in mice using PET scanning, gamma counter analysis of organ samples and high performance liquid chromatography (HPLC).


\section*{Key results}

The main result of the paper was that 18F-T807 labeling colocalised with PHF-tau-stained human brain sections in autoradiography images. Furthermore, the new tracer didn't colocalise with A$\beta$ plaques, suggesting the that 18F-T807 binds selectively to tau proteins and not to A$\beta$. These results were further confirmed by showing that 18F-T807 autoradiography signals correlate with PHF-tau loading and not with A$\beta$ loading. 

The tracer also displayed rapid brain penetration and fast washout in mice studies. Kidney elimiation was shown to be a significant clearance pathway, while activity in the bone and muscle remained relatively low.


% Don't mention about Kd, as they couldn't compare it to other values in the literature and that doesn't constitute a key result

\section*{Contributions}


\section*{Limitations}

% Did not test in other tauopaties


\section*{Impact and conclusion}



\nocite{*} % Show all Bib-entries
\bibliographystyle{unsrt}
\bibliography{citations}


\end{document}

