\documentclass[11pt,a4paper,oneside]{report}

\usepackage{amsmath,amssymb,calc,ifthen}

\usepackage[table,usenames,dvipsnames]{xcolor} % for coloured cells in tables

\usepackage{hyperref}


\usepackage{amsmath,graphicx}
\usepackage{epstopdf}
\usepackage{caption}
\usepackage{subcaption}

% highlight - useful for TODOs and similar
\usepackage{color}


\usepackage{listings}

\hypersetup{
  colorlinks,
  citecolor=Blue,
  linkcolor=Red,
  urlcolor=Blue}
  

% margin size
\usepackage[margin=2.5cm]{geometry}



\definecolor{mygreen}{rgb}{0,0.6,0}
\definecolor{mygray}{rgb}{0.5,0.5,0.5}
\definecolor{mymauve}{rgb}{0.58,0,0.82}


\begin{document}
\belowdisplayskip=12pt plus 3pt minus 9pt
\belowdisplayshortskip=7pt plus 3pt minus 4pt


\begin{titlepage}
\begin{center}

% Upper part of the page. The '~' is needed because \\
% only works if a paragraph has started.
\includegraphics[width=0.2\textwidth]{ucl-logo2}~\\[1cm]


\textsc{\Large CDT Journal Club - Second Written Review}\\[0.5cm]

\newcommand{\HRule}{\rule{\linewidth}{0.5mm}}

% Title
\HRule \\[0.4cm]
{ \Large [$^{18}$F]T807, a novel tau positron emission tomography imaging agent for Alzheimer's disease \\[0.4cm] }

{ \small Chun-Fang Xia, Janna Arteaga, Gang Chen, Umesh Gangadharmath, Luis F. Gomez, Dhanalakshmi Kasi, Chung Lam, Qianwa Liang, Changhui Liu, Vani P. Mocharla, Fanrong Mu, Anjana Sinha, Helen Su, A. Katrin Szardenings, Joseph C. Walsh, Eric Wang, Chul Yu, Wei Zhang, Tieming Zhao, Hartmuth C. Kolb }

\HRule \\[1.5cm]

% Author and supervisor
\begin{minipage}{0.4\textwidth}
\begin{flushleft} \large
\emph{Review Author:}\\
R\u{a}zvan Valentin \textsc{Marinescu}\\
\href{razvan.marinescu.14@ucl.ac.uk}{razvan.marinescu.14@ucl.ac.uk}\\
\end{flushleft}
\end{minipage}
\begin{minipage}{0.4\textwidth}
\begin{flushright} \large
\emph{Paper lead by:} \\
Prof. Erik \textsc{Arstad}\\
\end{flushright}
\end{minipage}

\vfill

EPSRC Centre for Doctoral Training in Medical Imaging\\ University College London

\vfill

% Bottom of the page
{\large \today}

\end{center}
\end{titlepage}
% \maketitle{}




\section*{Aims and motivations}

Tau-related neurological disorders such as Alzheimer's disease (AD), frontotemporal lobe degeneration (FTLD), Down's syndrome and progressive supranuclear palsy (PSP) affect more than 33.9 million people worldwide. \cite{barnes2011projected} Although many amyloid-beta (A$\beta$) tracers have been developed such as Pittsburgh compound B (PiB) \cite{klunk2004imaging}, only a few tracers have been reported previously for the tau protein, such as $^{18}$F-FDDNP\cite{barrio1999pet}, $^{11}$C-lansoprazole\cite{rojo2010selective} and $^{18}$F-THK523\cite{fodero201118f}. These had several limitations: $^{18}$F-FDDNP showed limited binding to both NFTs and A$\beta$ plaques in vitro \cite{thompson2009interaction}, $^{11}$C-lansoprazole uses carbon 11 ($^{11}$C) which has a shorter half life than fluorine 18 ($^{18}$F) and was initially unable to cross the brain-blood barrier \cite{shao2012evaluation} while THK523 was shown to bind to neurofibrillary tangles (NFTs) as well as to A$\beta$ plaques in human AD brains \cite{zeng20128invitro}. The aim of this paper by Xia et al. \cite{xia201318} was to design and develop a novel positron emission tomography (PET) imaging agent targeting paired-helical filament (PHF) tau in human AD brains that would overcome the limitations of previous compounds.

\section*{Methods}

The authors designed a novel class of 5H-pyrido-indoles and examined their binding affinity towards PHF tau and selectivity over A$\beta$. A total of 45 AD and non-AD human brains were immunohistochemically (IHC) stained for determining PHF-tau and A$\beta$ load. After a qualitative screening of several hundred compounds, T807 emerged as the lead candidate. Several preparation steps were required to label T807 with the $^{18}$F radioisotope by heating the mixture of the two compounds to 130 $^{\circ}$C so that the -NO$_2$ group would be substituted by $^{18}$F fluoride. In order to determine the lipophilicity of the new tracer, the logP was determined by analysing it in a gamma counter. 

The $^{18}$F-T807 tracer was then analysed with regards to binding selectivity to PHF tau over A$\beta$, brain uptake and biostability. Selectivity for PHF tau over A$\beta$ was measured using autoradiography on 26 AD and non-AD human brains. The dissociation constant $K_d$ was determined by Scatchard plot analysis. Brain uptake, biodistribution, excretion and biostability of $^{18}$F-T807 were analysed in mice using PET scanning, gamma counter analysis of organ samples and high performance liquid chromatography (HPLC).


\section*{Results}

The main result of the paper was that $^{18}$F-T807 labeling colocalised with PHF-tau-stained human brain sections in autoradiography images (fig. 3). Furthermore, the new tracer didn't colocalise with A$\beta$ plaques, suggesting the that $^{18}$F-T807 binds selectively to tau proteins and not to A$\beta$, having a selectivity bigger that 25 fold. These results were further confirmed by showing that $^{18}$F-T807 autoradiography signals correlate with PHF-tau loading and not with A$\beta$ loading (fig. 4). 

The tracer also displayed rapid brain penetration and fast washout in mice studies (section 3.5). Kidney elimiation was shown to be a significant clearance pathway, while activity in the bone and muscle remained relatively low. T807 also showed on-target binding after being assessed against a panel of 72 central nervous system (CNS) targets (section 3.8). The dissociation constant Kd was estimated at 14.6 nM, but this could not be compared against other literature values because of methodological differences.

\section*{Contributions}

The authors managed to develop a novel tau tracer that overcomes most of the limitations of previously reported compounds. $^{18}$F-T807 is a highly selective tau imaging agent, having a selectivity greater than 25 fold for PHF tau over A$\beta$. Mice experimens also showed that $^{18}$F-T807 can cross the brain-blood barrier efficiently and has a rapid washout. The use of Fluorine 18 ($^{18}$F) also ensures a long half life (110 minutes) necessary for monitoring long and complex processes in the human body. Another contribution is that the authors used a large number of brains (33 AD and 12 non-AD) and assessed binding potential of the imaging agent against a large number of CNS targets. 


\section*{Limitations}

% Did not test in other tauopaties
In my opinion, one of the main limitations of this paper is that the authors did not test the tracer in non-AD tauopathies such as Down's syndrome or frontotemporal lobar degeneration (FTLD), where different ultrastructural conformations of tau might be present. Moreover, as suggested by Villemagne et al. \cite{villemagne2014vivo}, further validation is necessary relating the binding of the tracer to clinical measures of cognitive impairment or to biomarkers such as gray matter atrophy, cerebrospinal fluid or plasma analytes known to be directly affected by tau deposition.  Another limitation is that the authors could not properly explain why $^{18}$F-T807 only binds to human PHF-tau and not to mouse PHF-tau. 


\section*{Impact and conclusion}

Given the favorable in-vivo properties of the new tau tracer, this work had a considerable impact in both the tau imaging and radiotracer design research areas. Although this work was published less than three years ago, it has already been cited by 37 papers and the novel tracer has been used in several human studies \cite{chien2012early,mitsis2014tauopathy}.  The first-in-human $^{18}$F-T807 study in AD, MCI and healthy participants showed that $^{18}$F-T807 follows the known distribution of PHF tau in the brain \cite{chien2012early}, where higher cortical retention of the agent was related to disease severity. \cite{villemagne2014vivo} The tracer has also been used to improve diagnosis of chronic traumatic encephalopathy \cite{mitsis2014tauopathy}, but in my opinion more research into non-AD tauopathies is required.

This work also helped develop $^{18}$F-T808 \cite{zhang2011highly}, which is a similar PET-based tau tracer designed by the authors at around the same time. Future development of selective tau imaging agents such as $^{18}$F-T807 will enable not only in vivo monitoring of tau deposition which is highly related to AD and non-AD cognitive decline, but also assessment of the relationship between tau and A$\beta$ \cite{villemagne2014vivo, fodero201118f, chien2012early, jack2010hypothetical}. I believe that the work of Xia et al. \cite{xia201318} is a clear step further in the design of tau-specific imaging agents that will help improve the diagnosis of AD and non-AD tauopathies and provide early detection in subjects at risk of developing these neurodegenerative disorders.


\nocite{*} % Show all Bib-entries
\bibliographystyle{unsrt}
\bibliography{citations}


\end{document}

